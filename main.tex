\documentclass{article}
\usepackage[utf8]{inputenc}
\usepackage{amsmath}
\usepackage[a4paper, total={6in, 8in}]{geometry}
\usepackage{graphicx}
%\usepackage{subfig} % use subfig instead
\usepackage{float}
\usepackage{amsmath}
\usepackage{amssymb}
\usepackage{subcaption}
\usepackage{blkarray}
\usepackage{multicol}
\usepackage{tabularx}
\usepackage[english]{babel}
\usepackage{blindtext}
\usepackage{cleveref}


%vector:
\newcommand{\vect}[1]{\ensuremath{\boldsymbol{\mathbf{#1} }}}
%matrix:
\newcommand{\matr}[1]{\ensuremath{\boldsymbol{\mathbf{#1} }}}
% Variance
\newcommand{\Var}{\textrm{Var}}
% Expectation
\newcommand{\E}{\textrm{E}}
% Covariance
\newcommand{\Cov}{\textrm{Cov}}
% Correlation
\newcommand{\Corr}{\textrm{Corr}}
% Standard Deviation
\newcommand{\Sd}{\textrm{SD}}

\newcommand{\domain}[1]{\ensuremath{\mathcal{#1}}}


\title{Assignment 2: Event Spatial Variables \\
\vspace{0.5cm}
\large{TMA4250 Spatial Statistics}
}

\author{Lene Tillerli Omdal, Arne Rustad}
\date{March 2021}

\begin{document}

\maketitle

\section*{Problem 1: Markov RF}

This problem is based on observations of seismic data over a domain $\domain D \in \mathbb{R}^2$, where the objective is to identify the underlying $\{sand, shale\}$ lithology distribution over $\domain D$, represented respectively by $\{0,1\}$.
The observations are on a regular $(75\times 75)$ grid $\domain L_D$, and the seismic data is denoted by $\{d(\vect x);\vect x \in \domain L_D\};d(\vect X) \in \mathbb{R}^2$, represented by the $n$-vector $\vect d$. Further the observations of the lithology distribution in a geologically comparable domain $ \domain D_c \subset \mathbb{R}^2$ is available. The distribution is collected on a regular $(66\time66)$ grid $\domain L_D_c$, with the same spacing as $\domain L_D$, over $\domain D_c$. We assume the underlying lithology distribution can be represented by a Mosaic RF $\{l(\vect x); \vect x \in \domain L_D\}; l(\vext x) \in \{0,1\} = \mathbb{L}$ represented by the $n$-vector $\vect l \in \mathbb{L}^n$. The space $\mathbb{L}^n$ is the space of all possible binary $n$-vectors. 

\subsection*{a)}

The seismic data collection procedure defines the likelihood model:

\begin{align*}
    [d_i|\vect l] = \begin{cases}0.02 +U_i & \text{if } l_i = 0 - \text{sand}\\
    0.08 +U_i & \text{if } l_i = 1 - \text{shale},
    \end{cases} \quad ; i = 1, 2, \dots , n
\end{align*}

with  $U_i; i = 1,2, \dots , n$ iid Gauss$\{0,0.06^2\}$. Since the observations are assumed to be conditionally independent from one node to another and have single-site response, then the likelihood model $p(\vect d| \vect l)$ on factorial form is given as

\begin{align*}
    [\vect d| \vect l] \sim p(\vect d| \vect l) &= \prod_{i=1}^n p(d_i|\vect l) = \prod_{i=1}^n p (d_i|l_i) = \prod_{i=1}^n \bigg( \phi_1(d_i;0.02, 0.06^2)I(l_i=0)+\phi_2(d_i;0.08,0.06^2)I(l_i=1)\bigg)\\
    &= \prod_{\substack{i=1 \\ l_i= 0}}^n \phi_1(d_i;0.02, 0.06^2)\prod_{\substack{i=1 \\ l_i = 1}}^n\phi_1(d_i;0.08,0.06^2).
\end{align*}
 

\subsection*{b)}

Consider a uniform, independence prior model on $\vect l$, i.e.

\begin{align*}
    p(\vect l) = const.
\end{align*}

From this we develop an expression for the posterior model $p(\vect l | \vect d$. This is given below

\begin{align*}
    [\vect l | \vect d] \sim p(\vect l | \vect d) &= \frac{p(\vect d| \vect l)p(\vect l)}{p(\vect d}\\
    &= \frac{\vect d |\vect l) p(\vect l)}{\sum_{\vect l' \in \mathbb{L}^n}p(\vect d| \vect l') p(\vect l')}\\
    &= \frac{p(\vect d| \vect l)}{\sum_{\vect l' \in \mathbb{L}^n} p (\vect d | \vect l')}\\
    &= \frac{\prod_{i=1}^n \bigg( \phi_1(d_i;0.02, 0.06^2)I(l_i=0)+\phi_2(d_i;0.08,0.06^2)I(l_i=1)\bigg)}{\sum_{\vect l' \in \mathbb{L}^n} \prod_{i=1}^n \bigg( \phi_1(d_i;0.02, 0.06^2)I(l'_i=0)+\phi_2(d_i;0.08,0.06^2)I(l'_i=1)\bigg)}.
\end{align*}

Then we simulate $6$ realizations of the posterior Mosaic RF $\{l(\vect x); \vect x \in \domain L_D| \vect D\}$. These are  displayed as maps in Figure \ref{}. 


\end{document}